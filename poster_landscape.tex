\documentclass[landscape,final]{baposter}

\usepackage{times}
\usepackage{calc}
\usepackage{graphicx}
\usepackage{amsmath}
\usepackage{amssymb}
\usepackage{relsize}
\usepackage{multirow}
\usepackage{bm}

\usepackage{graphicx}
\usepackage{multicol}
\usepackage{ulem}

\usepackage{pgfbaselayers}
\pgfdeclarelayer{background}
\pgfdeclarelayer{foreground}
\pgfsetlayers{background,main,foreground}

\usepackage{helvet}
%\usepackage{bookman}
\usepackage{palatino}

\newcommand{\captionfont}{\footnotesize}

\selectcolormodel{cmyk}

\graphicspath{{template_images/}}

%%%%%%%%%%%%%%%%%%%%%%%%%%%%%%%%%%%%%%%%%%%%%%%%%%%%%%%%%%%%%%%%%%%%%%%%%%%%%%%%
%%%% Some math symbols used in the text
%%%%%%%%%%%%%%%%%%%%%%%%%%%%%%%%%%%%%%%%%%%%%%%%%%%%%%%%%%%%%%%%%%%%%%%%%%%%%%%%
% Format 
\newcommand{\Matrix}[1]{\begin{bmatrix} #1 \end{bmatrix}}
\newcommand{\Vector}[1]{\Matrix{#1}}
\newcommand*{\SET}[1]  {\ensuremath{\mathcal{#1}}}
\newcommand*{\MAT}[1]  {\ensuremath{\mathbf{#1}}}
\newcommand*{\VEC}[1]  {\ensuremath{\bm{#1}}}
\newcommand*{\CONST}[1]{\ensuremath{\mathit{#1}}}
\newcommand*{\norm}[1]{\mathopen\| #1 \mathclose\|}% use instead of $\|x\|$
\newcommand*{\abs}[1]{\mathopen| #1 \mathclose|}% use instead of $\|x\|$
\newcommand*{\absLR}[1]{\left| #1 \right|}% use instead of $\|x\|$

\def\norm#1{\mathopen\| #1 \mathclose\|}% use instead of $\|x\|$
\newcommand{\normLR}[1]{\left\| #1 \right\|}% use instead of $\|x\|$

\def\half{{\textstyle{1\over2}}}
\def\d{{\rm d}}

%%%%%%%%%%%%%%%%%%%%%%%%%%%%%%%%%%%%%%%%%%%%%%%%%%%%%%%%%%%%%%%%%%%%%%%%%%%%%%%%
% Multicol Settings
%%%%%%%%%%%%%%%%%%%%%%%%%%%%%%%%%%%%%%%%%%%%%%%%%%%%%%%%%%%%%%%%%%%%%%%%%%%%%%%%
\setlength{\columnsep}{0.7em}
\setlength{\columnseprule}{0mm}


%%%%%%%%%%%%%%%%%%%%%%%%%%%%%%%%%%%%%%%%%%%%%%%%%%%%%%%%%%%%%%%%%%%%%%%%%%%%%%%%
% Save space in lists. Use this after the opening of the list
%%%%%%%%%%%%%%%%%%%%%%%%%%%%%%%%%%%%%%%%%%%%%%%%%%%%%%%%%%%%%%%%%%%%%%%%%%%%%%%%
\newcommand{\compresslist}{%
\setlength{\itemsep}{1pt}%
\setlength{\parskip}{0pt}%
\setlength{\parsep}{0pt}%
}


%%%%%%%%%%%%%%%%%%%%%%%%%%%%%%%%%%%%%%%%%%%%%%%%%%%%%%%%%%%%%%%%%%%%%%%%%%%%%%
%%% Begin of Document
%%%%%%%%%%%%%%%%%%%%%%%%%%%%%%%%%%%%%%%%%%%%%%%%%%%%%%%%%%%%%%%%%%%%%%%%%%%%%%

\begin{document}

%%%%%%%%%%%%%%%%%%%%%%%%%%%%%%%%%%%%%%%%%%%%%%%%%%%%%%%%%%%%%%%%%%%%%%%%%%%%%%
%%% Here starts the poster
%%%---------------------------------------------------------------------------
%%% Format it to your taste with the options
%%%%%%%%%%%%%%%%%%%%%%%%%%%%%%%%%%%%%%%%%%%%%%%%%%%%%%%%%%%%%%%%%%%%%%%%%%%%%%
\typeout{Poster Starts}
\background{
  \begin{tikzpicture}[remember picture,overlay]%
    \draw (current page.north west)+(-2em,-0em) node[anchor=north west] {\hspace{-2em}\includegraphics[height=1.1\textheight]{silhouettes_background}};
  \end{tikzpicture}%
}
\definecolor{silver}{cmyk}{0,0,0,0.3}
\definecolor{yellow}{cmyk}{0,0,0.9,0.0}
\definecolor{reddishyellow}{cmyk}{0,0.22,1.0,0.0}
\definecolor{black}{cmyk}{0,0,0.0,1.0}
\definecolor{darkYellow}{cmyk}{0,0,1.0,0.5}
\definecolor{darkSilver}{cmyk}{0,0,0,0.1}

\definecolor{lightyellow}{cmyk}{0,0,0.3,0.0}
\definecolor{lighteryellow}{cmyk}{0,0,0.1,0.0}
\definecolor{lighteryellow}{cmyk}{0,0,0.1,0.0}
\definecolor{lightestyellow}{cmyk}{0,0,0.05,0.0}
\begin{poster}{
  % Show grid to help with alignment
  grid=no,
  % Column spacing
  colspacing=1em,
  % Color style
  bgColorOne=lighteryellow,
  bgColorTwo=lightestyellow,
  borderColor=reddishyellow,
  headerColorOne=yellow,
  headerColorTwo=reddishyellow,
  headerFontColor=black,
  boxColorOne=lightyellow,
  boxColorTwo=lighteryellow,
  % Format of textbox
  textborder=roundedleft,
  % Format of text header
  eyecatcher=no,
  headerborder=open,
  headerheight=0.08\textheight,
  headershape=roundedright,
  headershade=plain,
  headerfont=\Large\textsf, %Sans Serif
  boxshade=plain,
%  background=shade-tb,
  background=plain,
  linewidth=2pt
  }
  % Eye Catcher
  {} % No eye catcher for this poster. If an eye catcher is present, the title is centered between eye-catcher and logo.
  % Title
  {\sf %Sans Serif
  %\bf% Serif
  High-order, adaptive methods for atomic structure calculations}
  % Authors
  {\sf %Sans Serif
  % Serif
  Ond\v rej \v Cert\'\i k$^{1,2}$, John Pask$^1$, Pavel \v Sol\'\i n$^2$
  \hspace{3em}$^1$ Lawrence Livermore National Laboratory,
  $^2$ University of Nevada, Reno
  }
  % University logo
  {{\begin{minipage}{16em}
    \hfill
    \includegraphics[height=5.5em]{llnl_logo}
  \end{minipage}}
  }

  \tikzstyle{light shaded}=[top color=baposterBGtwo!30!white,bottom color=baposterBGone!30!white,shading=axis,shading angle=30]

  % Width of left inset image
     \newlength{\leftimgwidth}
     \setlength{\leftimgwidth}{0.78em+8.0em}

%%%%%%%%%%%%%%%%%%%%%%%%%%%%%%%%%%%%%%%%%%%%%%%%%%%%%%%%%%%%%%%%%%%%%%%%%%%%%%
%%% Now define the boxes that make up the poster
%%%---------------------------------------------------------------------------
%%% Each box has a name and can be placed absolutely or relatively.
%%% The only inconvenience is that you can only specify a relative position 
%%% towards an already declared box. So if you have a box attached to the 
%%% bottom, one to the top and a third one which should be in between, you 
%%% have to specify the top and bottom boxes before you specify the middle 
%%% box.
%%%%%%%%%%%%%%%%%%%%%%%%%%%%%%%%%%%%%%%%%%%%%%%%%%%%%%%%%%%%%%%%%%%%%%%%%%%%%%
    %
    % A coloured circle useful as a bullet with an adjustably strong filling
    \newcommand{\colouredcircle}[1]{%
      \tikz{\useasboundingbox (-0.2em,-0.32em) rectangle(0.2em,0.32em); \draw[draw=black,fill=baposterBGone!80!black!#1!white,line width=0.03em] (0,0) circle(0.18em);}}

%%%%%%%%%%%%%%%%%%%%%%%%%%%%%%%%%%%%%%%%%%%%%%%%%%%%%%%%%%%%%%%%%%%%%%%%%%%%%%
  \headerbox{Abstract}{name=contribution,column=0,row=0}{
%%%%%%%%%%%%%%%%%%%%%%%%%%%%%%%%%%%%%%%%%%%%%%%%%%%%%%%%%%%%%%%%%%%%%%%%%%%%%%
   We compare several high-order methods for solving the radial Schr\"odinger
   and Dirac equations, in particulal spectral finite element method, $p$-FEM,
   $h$-FEM and $hp$-FEM. All methods provide robust way of calculating all
   eigenstates with any precision. They differ in the rate of convergence and
   whether a good initial mesh is needed or not, as well as providing robust
   error control.
 }

%%%%%%%%%%%%%%%%%%%%%%%%%%%%%%%%%%%%%%%%%%%%%%%%%%%%%%%%%%%%%%%%%%%%%%%%%%%%%%
  \headerbox{Motivation}{name=model,column=0,below=contribution}{
%%%%%%%%%%%%%%%%%%%%%%%%%%%%%%%%%%%%%%%%%%%%%%%%%%%%%%%%%%%%%%%%%%%%%%%%%%%%%%
The need for an efficient solution to the radial Dirac equation arises
in the calculation of the equation of state and opacity of materials
under extreme conditions.  These calculations often rely on
self-consistent average-atom codes to compute the atomic structure for a
representative atom in a plasma. For plasmas at low densities and high
temperatures, a very large number of Rydberg states are accessible,
often requiring the calculation of principal quantum numbers of 100 or
higher.  This poses a challenge for the existing average-atom models,
since they have difficultly in resolving all the bound states just below
the continuum, and accurately computing the wave-function of
high-principal-quantum-number states near the nucleus.
  }

%%%%%%%%%%%%%%%%%%%%%%%%%%%%%%%%%%%%%%%%%%%%%%%%%%%%%%%%%%%%%%%%%%%%%%%%%%%%%%
\headerbox{Schr\"odinger Equation}{name=fitting,column=0,below=model,above=bottom}{
%%%%%%%%%%%%%%%%%%%%%%%%%%%%%%%%%%%%%%%%%%%%%%%%%%%%%%%%%%%%%%%%%%%%%%%%%%%%%%
Radial Schr\"odinger equation:
$$(-\half \rho^2 R')' + (\rho^2 V + \half l(l+1)) R = \epsilon \rho^2 R$$
weak formulation
$$
\int \half \rho^2 R'v' + (\rho^2 V + \half l(l+1)) Rv \d\rho
        -\half[\rho^2R'v]_0^a =
$$
$$
= \epsilon \int \rho^2 Rv \d\rho
$$
}

%%%%%%%%%%%%%%%%%%%%%%%%%%%%%%%%%%%%%%%%%%%%%%%%%%%%%%%%%%%%%%%%%%%%%%%%%%%%%%
  \headerbox{Dirac Equation}{name=results neutralization,column=1,row=0}{
%%%%%%%%%%%%%%%%%%%%%%%%%%%%%%%%%%%%%%%%%%%%%%%%%%%%%%%%%%%%%%%%%%%%%%%%%%%%%%
Radial Dirac equations:
$$
-\hbar c \left({\d\over\d\rho} - {\kappa\over\rho}\right)Q + (V+mc^2-W)P=0
$$

$$
+\hbar c \left({\d\over\d\rho} + {\kappa\over\rho}\right)P + (V-mc^2-W)Q=0
$$
weak formulation:
$$
\int PVv_1 \d\rho + \int -\hbar c Q'v_1 + \hbar c{\kappa\over\rho}Qv_1 \d\rho =
\epsilon \int Pv_1 \d\rho
$$

$$
\int \hbar c P'v_2 + \hbar c{\kappa\over\rho}Pv_2 + \int (V -2mc^2)Qv_2 \d\rho =
\epsilon \int Q v_2 \d\rho
$$
  }
%%%%%%%%%%%%%%%%%%%%%%%%%%%%%%%%%%%%%%%%%%%%%%%%%%%%%%%%%%%%%%%%%%%%%%%%%%%%%%
  \headerbox{---\sout{Robustness}}{name=robustness,column=1,below=results neutralization,span=1,above=bottom}{
%%%%%%%%%%%%%%%%%%%%%%%%%%%%%%%%%%%%%%%%%%%%%%%%%%%%%%%%%%%%%%%%%%%%%%%%%%%%%%
  \begin{tabular}{@{}c@{ }c@{ }c@{ }c@{}}
    \includegraphics[height=0.42\linewidth]{56_4_tgt}&
    \includegraphics[height=0.42\linewidth]{23_2_tgt}&
    \includegraphics[height=0.42\linewidth]{5_6_tgt}\\[-0.8em]
                       & \smaller a) Targets & \\[0.8em]
    \includegraphics[height=0.42\linewidth]{56_4_expression}&
    \includegraphics[height=0.42\linewidth]{23_2_expression}& 
    \includegraphics[height=0.42\linewidth]{5_6_expression}\\[-0.8em]
                    & \smaller b) Fits & 
  \end{tabular}
  The reconstruction (b) is robust against scans (a) with artifacts, noise, and
  holes.
  }
%%%%%%%%%%%%%%%%%%%%%%%%%%%%%%%%%%%%%%%%%%%%%%%%%%%%%%%%%%%%%%%%%%%%%%%%%%%%%%
  \headerbox{---\sout{Results}}{name=results,column=2,span=2,row=0}{
%%%%%%%%%%%%%%%%%%%%%%%%%%%%%%%%%%%%%%%%%%%%%%%%%%%%%%%%%%%%%%%%%%%%%%%%%%%%%%
      \begin{multicols}{2}
        The method was evaluated on the GavabDB expression dataset which
        contains 427 Scans, with 3 neutral scans and 4 expression scans per ID.
        To test the impact of expression invariance on neutral data we used the
        UND Dataset from the Face Recognition Great Vendor Test, which contains
        953 neutral scans with one to eight scans per subject.
      \end{multicols}\vspace{-1em}
      \mbox{\hspace{0.3\linewidth}\rule{0.4\linewidth}{1pt}\hspace{0.3\linewidth}}\\
      \begin{tabular}{cc}
        \hspace{0.5em}\scalebox{0.74}{\input{shrec_MNCG}} &
        \hspace{0.5em}\scalebox{0.74}{\input{und_MNCG}}
      \end{tabular}\\
%      \begin{multicols}{2}
        {Expression neutralization improves results on the expression dataset
        without decreasing the accuracy on the neutral testset. Plotted is the
        ratio of correct answers to  the number of possible correct answers.
        %Note the different scales for the two graphs.
        %Our approach has a high accuracy on the neutral (UND) dataset.
        }
%      \end{multicols}\vspace{-1em}
      \\\mbox{\hspace{0.3\linewidth}\rule{0.4\linewidth}{1pt}\hspace{0.3\linewidth}}\\
      \begin{tabular}{cc}
        \hspace{0.5em}\scalebox{0.74}{\input{shrec_PR}} &
        \hspace{0.5em}\scalebox{0.74}{\input{und_PR}}
      \end{tabular}\\
%      \begin{multicols}{2}
        {Plotted are precision and recall for different retrieval depths. The lower
        precision of the UND database is due to the fact that some queries have no
        correct answers.}
%      \end{multicols}\vspace{-1em}
      \\\mbox{\hspace{0.3\linewidth}\rule{0.4\linewidth}{1pt}\hspace{0.3\linewidth}}\\
      \begin{tabular}{cc}
        \hspace{0.5em}\scalebox{0.74}{\input{shrec_FARFRR}} &
        \hspace{0.5em}\scalebox{0.74}{\input{und_FARFRR}}
      \end{tabular}\\
%      \begin{multicols}{2}
        {Impostor detection is reliable, as the minimum distance to a match
        is smaller than the minimum distance to a nonmatch. }
%      \end{multicols}
\\
  }%
%%%%%%%%%%%%%%%%%%%%%%%%%%%%%%%%%%%%%%%%%%%%%%%%%%%%%%%%%%%%%%%%%%%%%%%%%%%%%%
  \headerbox{---\sout{Open Questions}}{name=questions,column=2,span=1,above=bottom,below=results}{
%%%%%%%%%%%%%%%%%%%%%%%%%%%%%%%%%%%%%%%%%%%%%%%%%%%%%%%%%%%%%%%%%%%%%%%%%%%%%%
    While the expression and identity space are linearly independent, there is
    some expression left in the identity model. This is because a ``neutral''
    face is interpreted differently by the subjects. We investigate the
    possibilty to build an identity/expression separated model without using
    the data labelling, based on a measure of independence.
  }%
%%%%%%%%%%%%%%%%%%%%%%%%%%%%%%%%%%%%%%%%%%%%%%%%%%%%%%%%%%%%%%%%%%%%%%%%%%%%%%
  \headerbox{---\sout{Funding}}{name=funding,column=3,span=1,above=bottom}{
%%%%%%%%%%%%%%%%%%%%%%%%%%%%%%%%%%%%%%%%%%%%%%%%%%%%%%%%%%%%%%%%%%%%%%%%%%%%%%
  \smaller 
  This work was supported in part by Microsoft Research through the European PhD Scholarship Programme.
  }%
%%%%%%%%%%%%%%%%%%%%%%%%%%%%%%%%%%%%%%%%%%%%%%%%%%%%%%%%%%%%%%%%%%%%%%%%%%%%%%
  \headerbox{---\sout{References}}{name=references,column=3,above=funding,below=results}{
%%%%%%%%%%%%%%%%%%%%%%%%%%%%%%%%%%%%%%%%%%%%%%%%%%%%%%%%%%%%%%%%%%%%%%%%%%%%%%
    \smaller
    \vspace{-0.4em}
    \bibliographystyle{ieee}
    \renewcommand{\section}[2]{\vskip 0.05em}
      \begin{thebibliography}{1}\itemsep=-0.01em
      \setlength{\baselineskip}{0.4em}
      \bibitem{amberg07:nonrigid}
        B.~Amberg, S.~Romdhani, T. Vetter.
        \newblock {O}ptimal {S}tep {N}onrigid {ICP} {A}lgorithms for {S}urface {R}egistration
        \newblock In {\em CVPR 2007}
      \bibitem{amberg08:recognition}
        B.~Amberg, R.~Knothe, T. Vetter.
        \newblock Expression Invariant Face Recognition with a 3D Morphable Model
        \newblock In {\em AFGR 2008}
      \end{thebibliography}
  }%
\end{poster}%
%
\end{document}
